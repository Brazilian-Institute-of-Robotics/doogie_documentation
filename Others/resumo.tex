\begin{thesisresumo}
Estudos em Inteligência Artificial tem crescido e se diversificado em diferentes aplicações, e muitas vezes associadas à robótica, necessitando cada vez mais de profissionais capacitados para seu uso e exploração em diferentes ambientes. Por conta dessa necessidade, plataformas abertas de robótica podem ser uma potencial solução para o aprendizado introdutório dessas duas áreas. Portanto, o objetivo deste trabalho é desenvolver uma plataforma \textit{open source} baseada nos modelos utilizados na competição Micromouse. Essa plataforma será composta de um protótipo, ambiente de simulação e um labirinto para testes. Foi utilizada uma metodologia dividida em quatro partes: conceitual, design, desenvolvimento e conclusão. As fases dessa metodologia geraram materiais que culminaram na elaboração de dois protótipos com guias de montagem e configuração, ambiente de simulação utilizando ROS e Gazebo, proposta de labirinto modular  e um pacote de software para controle do robô. Por fim, é apresentado ao final do trabalho as conclusões obtidas, possíveis abordagens de uso da plataforma e trabalhos futuros.   

\ \\

% use de três a cinco palavras-chave

\textbf{Palavras-chave}: Inteligência Artificial, Robótica Móvel, Ensino de Robótica 

\end{thesisresumo}
