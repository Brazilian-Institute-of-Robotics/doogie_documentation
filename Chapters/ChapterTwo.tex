\chapter{Fundamentação Teórica}
\label{chap:fundteor}

\begin{flushright}

   \begin{list}{}{
      \setlength{\leftmargin}{4.5cm}
      \setlength{\rightmargin}{0cm}
      \setlength{\labelwidth}{0pt}
      \setlength{\labelsep}{\leftmargin}}
      \item Quanto maior for a rapidez de transformação de uma
      sociedade, mais temporárias são as necessidades
      individuais. Essas flutuaçõess tornam ainda mais acelerado
      o senso de turbilh da sociedade.

      \begin{list}{}{
      \setlength{\leftmargin}{0cm}
      \setlength{\rightmargin}{0cm}
      \setlength{\labelwidth}{0pt}
      \setlength{\labelsep}{\leftmargin}}
      \item (Alvin Toffler)
      \end{list}
   \end{list}
\end{flushright}

\begin{flushright}
  Quanto maior for a rapidez de transformação de uma \\
  sociedade, mais temporárias são as necessidades \\
  individuais. Essas flutuações tornam ainda mais \\
  acelerado o senso de turbilhão da sociedade. \\
  \ \\
  (Alvin Toffler)
\end{flushright}

%--------- NEW SECTION ----------------------
\section{Micromouse}
\label{sec:Micromouse}


%--------- NEW SECTION ----------------------
\section{Robotics Frameworks}
\label{sec:robotic_frameworks}



%--------- NEW SECTION ----------------------
\section{Estudo do estado da arte}
\label{sec:sota}
 \hspace{0.5cm} No cenário de desenvolvimento da robótica, faz-se necessário a assimilação de conceitos básicos para o encaminhamento dos projetos, seja em navegação, sensoriamento, visão computacional ou inteligência artificial. Por conta dessa necessidade, foi proposto o desenvolvimento de uma ferramenta que auxilie no processo de aprendizado, a partir do uso de uma plataforma físicas conhecida na robótica, o micromouse, que  faça uso de um framework de robótica para que se possibilite o desenvolvimento em alto nível. \\
 \hspace{0.5cm}  Esse tópico, portanto, objetiva trazer uma visão geral das tecnologias que estão sendo usadas no desenvolvimento de projetos micromouse, identificando as práticas relativas tanto a hardware quanto a software, bem como a disponibilidade da documentação e a finalidade diática desses projetos. 


%--------- NEW SECTION ----------------------
\section{Assunto 1}
\label{sec:ass1}
flkjasdlkfjasdlkfjs

%--------- NEW SECTION ----------------------
\section{Assunto 2}
\label{sec:ass2}
flkjasdlkfjasdlkfjs

