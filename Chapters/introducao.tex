\chapter{Introdução}
\label{chap:intro}

O mundo \'e - e sempre foi - um mundo de rede. Todavia apenas nas \'ultimas duas d\'ecadas a teoria de redes tornou-se um t\'opico que atraido aten\c{c}\~ao de pesquisadores e da m\'idia (refletida nos trabalhos de \cite{Barabasi2003-1}, \cite{Watts2003}, \cite{NBW2006}), especialmente em rela\c{c}\~ao \`as redes sociais: os relacionamentos entre os terroristas do 11/9, a forma como a SARS se espalhou em 2002/03 e o mito dos "6 graus de separa\c{c}\~ao" entre dois indiv\'iduos. At\'e mesmo a forma como a obesidade se espalha pode ser explicada atrav\'es da an\'alise de redes. O aumento da popularidade dos sites de rede social como Facebook, Google+ ou LinkedIn (ou a Plataforma Lattes brasileira) aumenta a nossa percep\c{c}\~ao de rede formada por nossos amigos, colegas e fam\'ilia e isso constitui a base invis\'ivel de nossa vida social.

%--------- NEW SECTION ----------------------
\section{Objetivo}
\label{sec:obj}

Nesta se\c{c}\~ao os objetivos principal (tamb\'em
pode-se se utilizar a palavra meta) da monografia de
gradua\c{c}\~ao ou especializa\c{c}\~ao, disserta\c{c}\~ao de
mestrado ou tese de doutorado s\~ao apresentados.


\subsection{Objetivos Específicos}
\label{ssec:objesp}

Nesta se\c{c}\~ao os objetivos espec\'ificos (tamb\'em
pode-se se utilizar a palavra meta) da monografia de
gradua\c{c}\~ao ou especializa\c{c}\~ao, disserta\c{c}\~ao de
mestrado ou tese de doutorado s\~ao apresentados.

%--------- NEW SECTION ----------------------
\section{Justificativa}
\label{sec:justificativa}
O cenário industrial mundial passou por grandes mudanças desde a primeira revolução industrial no final do século XVIII até os dias atuais. Atualmente, com o avanço da tecnologia e a sua inserção na indústria, uma nova revolução intitulada Indústria 4.0 está em andamento. Segundo a \citeonline[p. 11]{relatorio_cni}:
\begin{quoting}[rightmargin=0cm,leftmargin=4cm]
\begin{singlespace}
{\footnotesize
A incorporação da digitalização à atividade industrial resultou no conceito de Indústria 4.0, em referência ao que seria a 4ª revolução industrial, caracterizada pela integração e controle da produção a partir de sensores e equipamentos conectados em rede e da fusão do mundo real com o virtual, criando os chamados sistemas ciberfísicos e viabilizando o emprego da inteligência artificial.
}
\end{singlespace}
\end{quoting}

Essa nova revolução irá ocasionar impactos na economia nacional. Segundo levantamento da ABDI (Agência Brasileira de Desenvolvimento Industrial), o Brasil irá reduzir no mínimo R\$ 73 bilhões/ano, sendo que serão R\$ 34 bilhões com ganhos na eficiência no processo produtivo, R\$ 31 bilhões com redução de custos de manutenção de máquina e  R\$ 7 bilhões com economia de energia \cite{abdi}.

Destaca-se algumas tecnologias com maior influência nessa revolução tais como a internet das coisas (IoT, \textit{Internet of Things}), \textit{big data}, computação em nuvem, inteligência artificial e robótica avançada. Para \citeonline{industria_e_robo} o avanço da indústria em direção a fábricas mais inteligentes está vinculado com a evolução da automação, que devido a isso, torna a robótica um elemento crucial para a Indústria 4.0.

Para \citeonline{relatorio_cni} a consolidação da Indústria 4.0 no Brasil virá com possíveis consequências, dentre elas, o surgimento de novas atividades e novas profissões, que demandarão adptações no padrão de formação de recursos humanos.     

O pesquisador/estudante deve apresentar os aspectos mais
relevantes da pesquisa ressaltando os impactos (e.g. cient\'ifico,
tecnol\'ogico, econ\^omico, social e ambiental) que a pesquisa
causar\'a. Deve-se ter cuidado com a ingenuidade no momento em que
os argumentos forem apresentados.

%--------- NEW SECTION ----------------------
\begin{comment}
\section{Organização do trabalho}
\label{section:organizacao}

Este documento apresenta $5$ capítulos e está estruturado da seguinte forma:

\begin{itemize}

  \item \textbf{Capítulo \ref{chap:intro} - Introdução}: Contextualiza o âmbito, no qual a pesquisa proposta está inserida. Apresenta, portanto, a definição do problema, objetivos e justificativas da pesquisa e como este \thetypeworkthree está estruturado;
  \item \textbf{Capítulo \ref{chap:fundteor} - Fundamentação Teórica}: XXX;
  \item \textbf{Capítulo \ref{chap:mat} - Materiais e Métodos}: XXX;
  \item \textbf{Capítulo \ref{chap:result} - Resultados}: XXX;
  \item \textbf{Capítulo \ref{chap:conc} - Conclusão}: Apresenta as conclusóes, contribuições e algumas sugestões de atividades de pesquisa a serem desenvolvidas no futuro.

\end{itemize}
\end{comment}